%%%%%PLEASE CONSIDER CORRECTIONS AT PLACES INDICATED%%%%%%%%
\documentclass{article}
%%%%%Packages Used, add more if necessary%%%%
\usepackage{amsmath,amsfonts,amssymb,graphicx,fullpage}
\setlength{\topmargin}{-0.6 in}
\setlength{\textheight}{8.5 in}
\setlength{\headsep}{0.75 in}



%%%%% NO NEED TO EDIT THIS PREAMBLE %%%%%%%%
%%% PREAMBLE %%%
\newcounter{lecnum}
\renewcommand{\thepage}{\thelecnum-\arabic{page}}
\renewcommand{\thesection}{\thelecnum.\arabic{section}}
\renewcommand{\theequation}{\thelecnum.\arabic{equation}}
\renewcommand{\thefigure}{\thelecnum.\arabic{figure}}
\renewcommand{\thetable}{\thelecnum.\arabic{table}}
\newcommand{\lecture}[4]{
   \pagestyle{myheadings}
   \thispagestyle{plain}
   \newpage
   \setcounter{lecnum}{#1}
   \setcounter{page}{1}
   \noindent
   \begin{center}
   \framebox{
      \vbox{\vspace{2mm}
    \hbox to 6.28in { {\bf CS 7545: Machine Learning Theory
		\hfill Fall 2019} }
       \vspace{4mm}
       \hbox to 6.28in { {\Large \hfill Lecture #1: #2  \hfill} }
       \vspace{2mm}
       \hbox to 6.28in { {\it Lecturer: #3 \hfill Scribes: #4} }
      \vspace{2mm}}
   }
   \end{center}
   \markboth{Lecture #1: #2}{Lecture #1: #2}


   {\bf Disclaimer}: {\it These notes have not been subjected to the
   usual scrutiny reserved for formal publications.}
   \vspace*{4mm}
}
\renewcommand{\cite}[1]{[#1]}
\def\beginrefs{\begin{list}
        {[\arabic{equation}]}{\usecounter{equation}
         \setlength{\leftmargin}{2.0truecm}\setlength{\labelsep}{0.4truecm}%
         \setlength{\labelwidth}{1.6truecm}}}
\def\endrefs{\end{list}}
\def\bibentry#1{\item[\hbox{[#1]}]}

\newcommand{\challenge}[2]{\noindent \textbf{(Challenge Problem)} \emph{#1}: #2 }

\newcommand{\exercise}[1]{\noindent \textbf{(Exercise)} #1 }

\newcommand{\fig}[3]{
			\vspace{#2}
			\begin{center}
			Figure \thelecnum.#1:~#3
			\end{center}
	}
%%% END_OF_PREAMBLE %%%


	
%%%%You may add more \newtheorem if necessary%%%%%%
\newtheorem{theorem}{Theorem}[lecnum]
\newtheorem{lemma}[theorem]{Lemma}
\newtheorem{proposition}[theorem]{Proposition}
\newtheorem{claim}[theorem]{Claim}
\newtheorem{corollary}[theorem]{Corollary}
\newtheorem{definition}[theorem]{Definition}
\newenvironment{proof}{{\bf Proof:}}{\hfill\rule{2mm}{2mm}}


\newcommand{\dom}{\mathrm{dom}}
\begin{document}

%%%%%CHANGE HERE%%%%%%%
%%%%%\section{title of the section} similarly with the rest \section{} or \subsection{} or \subsubsection{} etc


%%%%%CHANGE HERE%%%%%%%%%
%%%%%\lecture{the ordinal number of the lecture}{lecture title}{Jacob Abernethy}{scriber's name}%%%%%%%%
\lecture{1}{TITLE OF LECTURE}{Jacob Abernethy}{NAME OF SCRIBE(S)}

\section{Style Guide}
\paragraph{Notations}

\begin{itemize}
  \item Use \verb|\vec{x}| to denote an $n$-dimensional vector $\vec{x}$, and $x_i$ (not boldfaced) to denote its $i$-th coordinate of $\vec{x}$.
  \item Use an uppercase letter to denote a matrix: $M$.
  \item Use \verb|\top| for transposition: $\vec{x}^\top M \vec{x}$.
  \item Define new math operators in the preamble: \verb|\newcommand{\dom}{\mathrm{dom}}| and then use $\dom(f)$.
\end{itemize}

\paragraph{Styles}

\begin{itemize}
    \item Please avoid using logic symbols (such as $\forall$, $\iff$, and $\implies$) as a part of your English sentence when you present a definition, statement (claim/proposition/theorem/lemma), or proof. Feel free to use them in the less ``formal'' settings.
  \item In a definition block, boldface the concept being defined. \end{itemize}
Example:

\begin{definition}[strongly convex]
A differentiable function $f$ is \textbf{$c$-strongly convex} if for all $\vec{x},\vec{y} \in \dom(f)$, 
\[f(\vec{y}) \ge f(\vec{x}) + \nabla f(\vec{x})^T (\vec{y} - \vec{x}) + \frac{c}{2} \|\vec{y} - \vec{x}\|^2.\]
\end{definition}
%%%%%CHANGE HERE%%%%%%%
%%%%%\section{title of the section} similarly with the rest \section{} or \subsection{} or \subsubsection{} etc
\section{Title of the first section of the lecture} 
%%%%%Use itemize to layout bullet points that are not numbered%%%%%

The professor said a few things. Then he wrote this on the board. This was followed by a list of things.
\begin{itemize}
\item This is the first thing in the list
\item The following is a true statment: $1 + 1 = 2$
\item This sentence is false. (Or is it?)
\item Now I'm going to use new notation. Let's let $n$ be an integer. Then notice that
\[
  \sum_{i=1}^\infty 1[n \geq i] = n
\]
where $1[]$ is the indicator function\footnote{I just used some new notation so I should define it here:
$
1[\textnormal{statement}] = \begin{cases}

  1 & \mbox{if statement true;} \\

  0 & \mbox{if statement false} 
  \end{cases}
  $.}
\end{itemize}

\subsection{Some special commands}

\exercise{The professor mentioned an exercise in class that would be useful to work out. You can use the \texttt{$\backslash$exercise} command in these cases.}

\challenge{Name of Challenge Problem}{If a challenge problem is given out, give it a name and put it in the previous field and then write down the description in this field.}

\section{Theorems etc.}

\begin{lemma} \label{lem:biglemma}
  We are going to need this result in a moment.
\end{lemma}
\begin{proof}
  I'm now proving the lemma.
\end{proof}

\begin{theorem}
  Here's a big statement
\end{theorem}
\begin{proof}
  Here's a proof of this big statement. It basically follows from Lemma~\ref{lem:biglemma}.  
\end{proof}

\begin{corollary}
  This follows from the theorem, I swear.
\end{corollary}


\end{document}
